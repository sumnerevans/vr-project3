\documentclass[11pt]{article}
\usepackage[margin=1in]{geometry}
\usepackage{enumitem}
\usepackage{minted}
\usepackage{titlesec}
\usepackage{textcomp}
\usepackage{amsmath}
\usepackage[makeroom]{cancel}

\setlength{\parindent}{0cm}

\title{Project 3 Retrospective}
\author{Sumner Evans}
\date{\today}

\begin{document}
\maketitle

\section{Introduction}
For my individual project, I created Snowflakes, a winter-themed environment
where users can create their own structures by manipulating snow blocks.
Although I did not accomplish all of my initial goals, the project was extremely
successful and I learned valuable lessons about user interaction via the
controllers and handling physics-aware VR objects.

\section{Goals}
The following list describes my initial goals for Snowflakes. (Items marked with
an * indicate that the goal was initially a stretch goal.)
\begin{itemize}
    \item The environment will have ``snow'' on the ground.
    \item The environment will have static winter themed items.
    \item The environment will have snow falling.
    \item The user will be able to create new snow blocks by pointing both
        controllers at the ground and pulling both of the triggers. This action
        will create a snow block being held between the users hands.
    \item When the user releases the triggers, the snow block will fall.
    \item The user will be able to grab existing blocks by pointing at them with
        both controllers and pulling both triggers. This will teleport the block
        into their hands.
    \item The snow blocks will fall in a physically-accurate manner with
        collision detection.
    \item * The user will be able to create snowballs by pointing one
        controller at the ground and pulling the trigger.
    \item * The user will be able to throw the snowballs at the blocks to
        ``break'' the blocks.
    \item * Sounds such as Christmas music will be output to the VR headset's
        audio jack to enhance the experience.
\end{itemize}

\section{Changes for the Actual Implementation}
\begin{itemize}
    \item I did not add snow to the ground. This proved to be problematic since
        we have not yet implemented shadows for \texttt{flight}. The actual snow
        blocks would have been too hard to see.

    \item I did not add snowflakes falling from the sky because \texttt{flight}
        does not yet have good support for rendering particle systems and I did
        not have time (or expertise) to implement that functionality myself.
\end{itemize}

\section{Unexpected Roadblocks}
I ran into many unexpected difficulties while implementing this project. Below
are three that stood out as most difficult to solve.
\begin{itemize}
    \item Integrating the \texttt{nphysics3d} library with existing items in the
        world proved to be difficult. I had to associate the physics world and
        the rendering world in a way that the rendering world would be updated
        when the physics world changed.

    \item Dealing with time step of the physics world was non-trivial. After
        much debugging, I realized that, if the world time-step was too large,
        collision handling started to behave oddly. From the time of the
        application start to the first frame can be close to 2 seconds which
        leads to problems to the physics library. To fix this, I maxed out the
        time step to a small number.

    \item Pointing at things in the world ended up being much more difficult
        than I initially expected. The largest difficulty was that with the
        libraries I used, static objects in the world (such as the floor) do not
        participate in certain collision detection functions. I ended up having
        to use some dot-product math to determine if the user was pointing to
        the floor or not when the pulled the triggers.
\end{itemize}

\section{Conclusion}
I was unable to implement all of my initial goals for Snowflakes, but I was able
to implement the core functionality of creating snow blocks and having them
interact in a physically-accurate manner. Overall, the project was a success and
I look forward to improving it for the final deliverable.

\end{document}
